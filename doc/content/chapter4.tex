
\chapter{Applied mathematics}

This chapter describes the usage of KamilaLisp to achieve various results in applied mathematics.

\section{Polynomials}

KamilaLisp does not support a native polynomial data type, however, throughout the section a \textit{litte-endian} representation is assumed, such that for instance the polynomial $P(x) = 2x^2 + 3x + 1$ is represented as the list $(1, 3, 2)$.

\subsection{Numerically evaluating rational sums and integrals}

Suppose that we want to compute the values of the following expressions, given some polynomial functions $P(x)$ and $Q(x)$:

$$
\sum_{i=1}^\infty \frac{P(i)}{Q(i)}
$$

$$
\int_1^\infty \frac{P(x)}{Q(x)} dx
$$

\noindent The problem is that these expressions may not converge, or may converge very slowly. Precisely, $\deg P(x) - \deg Q(x) < -1$ is a sufficient condition for divergence. To demonstrate a particular case of performing the summation, consider the following:

$$\sum_{k=1}^\infty \frac{1}{k^2 - a}$$

\noindent where $a$ is a constant. The following equalities hold (the first equality: only for $\sqrt{a} \in \mathbb{C} \setminus \mathbb{Z}$):

$$\sum_{k=1}^\infty \frac{1}{k^2 - a} = \frac{1-\sqrt{a}\pi\cot(\sqrt{a}\pi)}{2a} = \frac{1}{2\sqrt{a}}\left(\psi\left(1+\sqrt{a}\right)-\psi\left(1-\sqrt{a}\right)\right)$$

\noindent To attempt at understanding why is this happening, recall the following identity due to Euler (1770) for $z \in \mathbb{C} \setminus \mathbb{Z}$:

$$
\pi \cot(\pi z) = z^{-1} + \sum_{v=1}^\infty \frac{2z}{z^2 - v^2}
$$

\noindent A concise proof using Mittag-Leffler's theorem follows. Given $b_n$ as the residues:

$$
f(z) = f(0) + \sum_{n=1}^\infty b_n \left(\frac{1}{z-z_n} + \frac{1}{z_n}\right) = f(0) + \sum_{n=1}^\infty \frac{zb_n}{z_n(z_n-z)}
$$

\noindent Using the contour integral where $C_N$ is a circle enclosing first $N$ poles of $f$:

$$
I_N = \oint_{C_N} \frac{f(\omega)d\omega}{\omega(\omega - z)}
$$

\noindent Using the residue theorem we find:

$$
I_N = -2\pi i \frac{f(0)}{f(z)} + 2\pi i \frac{f(z)}{z} + \sum_{n=1}^N \frac{2\pi i b_n}{z_n(z_n - z)}
$$

\noindent Consider $\cot z - z^{-1}$ to remove a singularity. $b_n$ is found using the residue theorem as follows:

$$
b_n = \lim_{z \to n\pi} (z-n\pi)\cot z = \lim_{z \to n\pi} (z-n\pi) \frac{z \cos z - \sin z}{z \sin z} = 1
$$

\noindent Hence:

$$
\cot z - z^{-1} = \sum_{n=1}^N \frac{1}{z-n\pi} + \frac{1}{n\pi} + \frac{1}{z+n\pi} - \frac{1}{n\pi}
$$

\noindent Substitute and rearrange:

$$
\pi \cot(\pi z) = z^{-1} + \sum_{n=1}^N \left(\frac{1}{z-n} + \frac{1}{z+n}\right) = z^{-1} + \sum_{n=1}^N \frac{2z}{z^2-n^2}
$$

\noindent Knowing where the first part of the equality comes from, the next logical step is reasoning about the second equality:

$$
\frac{1-\sqrt{a}\pi\cot(\sqrt{a}\pi)}{2a} = \frac{1}{2\sqrt{a}}\left(\psi\left(1+\sqrt{a}\right)-\psi\left(1-\sqrt{a}\right)\right)
$$

\noindent Use the recurrence relation of the digamma function once:

$$
\frac{1-\sqrt{a}\pi\cot(\sqrt{a}\pi)}{2a} = \frac{1}{2\sqrt{a}}\left(\frac{1}{\sqrt{a}}+\psi\left(\sqrt{a}\right)-\psi\left(1-\sqrt{a}\right)\right)
$$

\noindent Now it is possible to apply the digamma reflection formula given as follows:

$$
\psi (1-x)-\psi (x)=\pi \cot \pi x
$$

\noindent The problem hence reduces to:

$$
\frac{1}{2\sqrt{a}}\left(\frac{1}{\sqrt{a}}-\pi\cot\pi\sqrt{a}\right) = \frac{1}{2\sqrt{a}}\left(\frac{1-\sqrt{a}\pi\cot\pi\sqrt{a}}{\sqrt{a}}\right) = \frac{1-\sqrt{a}\pi\cot\pi\sqrt{a}}{2a}
$$

The basic algorithm for convergent series implemented below is as follows. Consider polynomials $P(x)$ and $Q(x)$ that have only unique roots (i.e. the multiplicity of each is 1). Denote the roots of the polynomial $Q(x+1)$ (obtained using the most convenient means) as $\omega_0 \dots \Omega_m$. The following equality holds if $P(x)=1$:

$$
\sum_{n=1}^\infty \frac{1}{Q(n)} = -\sum_{\omega \in R} \frac{\psi(-\omega)}{Q'(\omega + 1)}
$$

If $P(x) \ne 1$ (i.e. the rational function is not an inverse of some polynomial), the formula is as follows:

$$
\sum_{n=1}^\infty \frac{P(n)}{Q(n)} = -\sum_{\omega \in R} \frac{P(\omega + 1) \psi(-\omega)}{Q'(\omega + 1)}
$$

To start, define a few helper functions - evaluating a polynomial at a point, computing the symbolic derivative of a polynomial and substituting $x+1$ into the polynomial to transform the \textit{litte-endian} coefficient list:

\begin{Verbatim}
    (def polyevl
      [(inner-product cmplx64:+ cmplx64:*) #0 ^cmplx64:**&[#0 range@tally]])

    (def polyderv cdr@[* #0 range@tally])

    (defun poly+1 (c)
      (let-seq
        (def bin-mat (:[:^binomial #0 \range $(+ 1)]@range@tally c))
        (def coeff-mat (:$(take (tally c)) (* c bin-mat)))
        (:$(foldl1 +)@transpose coeff-mat)))
\end{Verbatim}

The function is trivially defined as follows:

\begin{Verbatim}
    (defun rsum (p q) (let-seq
      (def q+1 (poly+1 q))
      (def p+1 $(polyevl (poly+1 p)))
      (def q+1p $(polyevl (polyderv q+1)))
      (def R (cmplx64:solve q+1))
      (def V [/ [* p+1 cmplx64:digamma@-] q+1p])
      (cmplx64:neg (foldl + 0 (:V R)))))
\end{Verbatim}

The integral case is defined similarly; except the digamma factor is replaced with a logarithm factor such that we sum $\displaystyle \frac{P(w) \log(x-w)}{Q'(w)}$ for roots $w$; the \verb|rat-int| function returns the numeric antiderivative of the specified rational function. 

\begin{Verbatim}
    (defun rat-int (p q)
      (let-seq
        (def qp \polyderv q)
        (def qr \cmplx64:solve q)
        (def q' \:$(polyevl qp) qr)
        (λ x \-@foldl + 0 \^/ q' \:(λ r \^polyevl r \* p \ln@- x r) qr)))
\end{Verbatim}

Using \verb|rsum| we estimate the value of $\displaystyle \sum_{k=1}^{\infty} \frac{1}{k^2-2}$ as follows:

\begin{Verbatim}
    --> rsum '(1) '(-2 0 1)
    -0.056810407700620236J4.879779454488352E-17
\end{Verbatim}

The result is correct to approximately 17 decimal places, as seen by the imaginary part of the result, which in reality is zero. To compute the value of this sum, use the previously proven identity:

$$\sum_{k=1}^\infty \frac{1}{k^2 - 2} = \frac{1-\sqrt{2}\pi\cot(\sqrt{2}\pi)}{4} = \frac{1}{2\sqrt{2}}\left(\psi\left(1+\sqrt{2}\right)-\psi\left(1-\sqrt{2}\right)\right)$$

Verifying the result using KamilaLisp yields:

\begin{Verbatim}
    --> flt64:* 0.25 (flt64:- 1 ([flt64:* #0 flt64:cot] (flt64:pi@flt64:sqrt 2)))
    -0.056810407700620236

    --> (flt64:/
    ...   (foldl1 - (:flt64:digamma ([tie flt64:+ flt64:-] 1 (flt64:sqrt 2))))
    ...   ([flt64:* #0 flt64:sqrt] 2))
    -0.05681040770062016
\end{Verbatim}

The general formula in the case when the denominator has a root of multiplicity greater than one is taken care of as follows by performing partial fraction decomposition in the complex field:

$$
\sum_{n=0}^{\infty }\frac{P(n)}{Q(n)}=\sum _{n=0}^{\infty }\sum
_{k=1}^{m}{\frac {a_{k}}{(n+b_{k})^{r_{k}}}}=\sum _{k=1}^{m}{\frac
{(-1)^{r_{k}}}{(r_{k}-1)!}}a_{k}\psi ^{(r_{k}-1)}(b_{k})
$$

The basic method presented above is a generalisation of the Heaviside cover-up method. Consider the following expansion of $\frac{P(x)}{Q(x)}$:

$$
\frac{P(x)}{Q(x)}=\sum _{i}\left({\frac {a_{i1}}{x-x_{i}}}+{\frac {a_{i2}}{(x-x_{i})^{2}}}+\cdots +{\frac {a_{ik_{i}}}{(x-x_{i})^{k_{i}}}}\right)
$$

Further, let:

$$
g_{ij}(x)=(x-x_{i})^{j-1}\frac{P(x)}{Q(x)}
$$

According to the uniqueness of Laurent series, $a_{ij}$ is the coefficient of the term $(x − x_i)^{−1}$ in the Laurent expansion of $g_ij(x)$ about the point $x_i$, i.e., its residue ${\displaystyle a_{ij}=\operatorname {Res} (g_{ij},x_{i})}$. In the general case, $a_{ij}$ is given by:

$$
{\displaystyle a_{ij}={\frac {1}{(k_{i}-j)!}}\lim _{x\to x_{i}}{\frac {d^{k_{i}-j}}{dx^{k_{i}-j}}}\left((x-x_{i})^{k_{i}}\frac{P(x)}{Q(x)}\right)}
$$

In the special case when $x_i$ is a simple root (i.e. has the multiplicity of one, as assumed in the basic scenario):

$$
a_{i1}={\frac {P(x_{i})}{Q'(x_{i})}}
$$

\subsection{Polynomial discriminants}

Let $P(x)$ denote general monic polynomial of degree $n$ i.e. of the form

$$
P(x) = (x-x_1)(x-x_2) \cdots (x-x_n)
$$

\noindent Let $V(x_1,x_2, \cdots, x_n)$ denote the Vandermonde deteminant in $x_1,x_2, \cdots, x_n$

$$
V(x_1,x_2, \cdots , x_n) = \prod\limits_{1 \leq i < j \leq n} (x_j - x_i)
$$

\noindent The discriminant of $P(x)$ is denoted by $\Delta_P$ and it is defined as:

$$
\Delta_P = V(x_1,x_2, \cdots , x_n)^2 = \prod\limits_{1 \leq i < j \leq n} (x_j - x_i)^2
$$

The algorithm starts by computing the cartesian product of the roots of the polynomial and subtracting them. Then, a triangular matrix is built as a mask for the product of the roots. After multiplying the product and the mask, the result is flattened and multiplied together:

\begin{Verbatim}
    --> defun Δ (p) (re (foldl1 * (flatten@replicate
    ...   ([(outer-product <) #0 #0] (range@tally p))
    ...   (** (outer-product - p p) 2))))
\end{Verbatim}

\section{Integer arithmetic coding}

Integer arithmetic coding is a form of entropy encoding used in lossless data compression. A string of characters is usually represented using a fixed number of bits per character (e.g. using ASCII). When a string is compressed using arithmetic coding, frequently used characters will be stored using fewer bits and less frequently occurring characters will be stored with more bits, resulting in fewer bits used in total. Unlike Huffman coding presented previously in the book, integer arithmetic coding is optimal.

The algorithm behind integer arithmetic coding is as follows:

\begin{itemize}
  \item Pick a large integer which is a multiple of 4 and call it $M$ (the maximum).
  \item Define the interval $[L, H) = [0, M)$ and $k=0$ (the underflow count).
  \item Perform underflow expansion: While $M/4 \leq L \leq M/2 \leq H \leq 3M/4$, update the interval by reassigning $L$ to $2L-M/2$ and $H$ to $2H-M/2$.
  \item Perform rescaling: If $H \le M/2$, then double $L$ and $H$, add $01^k$ to the code word, reset $k$ to 0 and go back to the underflow expansion step.
  \item Perform rescaling: If $L \ge M/2$, then update the interval by reassigning $L$ as $2L-M$ and $H$ as $2H-M$. Add $10^k$ to the code word, reset $k$ to 0 and go back to the underflow expansion step.
  \item If all symbols are encoded:
  \begin{itemize}
    \item If $L < M/4$ add $01^{k+1}$ to the code word.
    \item Otherwise add $10^{k+1}$ to the code word.
  \end{itemize}
  and terminate.
  \item Encode the symbol by reassigning the range as follows:
  $$
  [L, H) \Rightarrow \left[L+\left\lfloor\sum_{j=1}^{i-1} p_j(H-L)\right\rfloor, L+\left\lfloor\sum_{j=1}^{i} p_j(H-L)\right\rfloor\right]
  $$
\end{itemize}

Notice that due to the behaviour of the floor function, the following situation may occur:

$$
\left\lfloor\sum_{j=1}^{i-1} p_j(H-L)\right\rfloor = \left\lfloor\sum_{j=1}^{i} p_j(H-L)\right\rfloor
$$

This situation is prevented by approperiately choosing $M$, such that it is divisible by 4. Further, the following must hold:

$$
M \geq \frac{4}{p_\text{min}} - 8 \text{  , where  } p_\text{min} = \min_{i} p_i
$$

A KamilaLisp implementation (obtained by directly translating the procedure above) follows:

\begin{Verbatim}
(defun ari-enc (b M p) ((lambda (b M L H k p)
  ; Step 1: Underflow Expansion
  (if (and (<= (/ M 4) L) (< L (/ M 2)) (< (/ M 2) H) (<= H (/ (* 3 M) 4)))
    (&0 b M (- (* 2 L) (/ M 2)) (- (* 2 H) (/ M 2)) (+ k 1) p)
  ; Step 2: Rescaling (pt 1.)
  (if (<= H (/ M 2))
    (append (cons 0 (replicate k 1)) (&0 b M (* 2 L) (* 2 H) 0 p))
  ; Step 3: Rescaling (pt 2.)
  (if (>= L (/ M 2))
    (append (cons 1 (replicate k 0)) (&0 b M (- (* 2 L) M) (- (* 2 H) M) 0 p))
  ; Step 4: All symbols encoded?
  (if (empty? b)
    (if (<= L (/ M 4))
      (cons 0 (replicate (+ k 1) 1)) ; L <= M/4
      (cons 1 (replicate (+ k 1) 0))) ; H >= 3M/4
  (let ((HL (- H L)) (h (car b)))
    (&0 (cdr b) M
      (+ L (floor (foldl + 0 (* HL (take h p)))))
      (+ L (floor (foldl + 0 (* HL (take (+ h 1) p)))))
      k p))))))) b M 0 M 0 p))
\end{Verbatim}

Consider the alphabet $\Sigma = \{\text{A}, \text{B}, \text{EOF}\}$ with the probabilities $p_\text{A} = 0.6$, $p_\text{B} = 0.3$ and $p_\text{EOF} = 0.1$. The message $\text{A B A EOF}$ is thus encoded considering $M = 16$ as follows:

\begin{Verbatim}
    --> ari-enc '(0 1 0 2) 16 '(0.6 0.3 0.1)
    (0 1 1 0 1 0 0 1)
\end{Verbatim}

\section{Finite-precision arithmetic}

KamilaLisp offers a wide range of finite precision (\verb|flt64| and \verb|cmplx64|) functions. Outside of the functions that correspond to existing, arbitrary precision primitives (e.g. \verb|log2|, \verb|sqrt|, \verb|**|, \verb|sin|), KamilaLisp also offers the evaluation of various special functions in the finite precision domain. 
